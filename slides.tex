\documentclass[aspectratio=169]{beamer}

\usepackage{./mpidrtheme/mpidr21}

%%this package is only here to include latex-sourcecode in the example-presentation
\usepackage{verbatim}
\usepackage{natbib}
\usepackage{fontawesome5}
\usepackage{natbib}
\usepackage{apalike}
\usepackage{amssymb}
\usepackage{algorithm}
\usepackage{babel}
\newcommand\boldgreen[1]{\textcolor{mpidrgreen}{\textbf{#1}}}
\AtBeginSection[]
{
\begin{frame}{Ablauf}
\tableofcontents[currentsection]
\end{frame}
}
% Declare the math size
\DeclareMathSizes{50}{30}{10}{10}

% Create the colours
\definecolor{mpidr_red}{RGB}{180, 55, 70}
\definecolor{mpidr_blue}{RGB}{14, 123, 172}
\definecolor{mpidr_green}{RGB}{6, 110, 110}
\definecolor{mpidr_orange}{RGB}{200, 125, 0}
\definecolor{mpidr_grey}{RGB}{100, 100, 100}



\bibliographystyle{kluwer}
\begin{document}

%%%%%%%%%%%%%%%%%%%%%%%%%%%%%%%%%%%%%%%%%%%%%%%%%%%%%%%%%%%%%%%%%%%%%%%
%%	title page information
%%%%%%%%%%%%%%%%%%%%%%%%%%%%%%%%%%%%%%%%%%%%%%%%%%%%%%%%%%%%%%%%%%%%%%%
\title[Theile]									
{Demografie im Wandel: \\
\small Wie sich die Bev\"olkerung von Rostock verändert}
\author[Theile]{Henrik-Alexander Schubert\inst{1}}

\institute[MPIDR]{\inst{1}Doktorand am \textit{Max-Planck-Institut für Demografische Forschung} und der \textit{University of Oxford}}

%\Email{theile@mpidr.de}

\begin{frame}[mpidrbackground=1]
  \titlepage
\end{frame}


% Wimmelbild
\begin{frame}[mpidrbackground = 4]
    \frametitle{Was ist die Hanse- \& Universit\"atsstadt Rostock f\"ur dich?}
    \begin{figure}
   \includegraphics[width = \textwidth]{figures/rostock_wimmelbild.pdf}
\end{figure}
\end{frame}

% Hansa
\begin{frame}[mpidrbackground = 4]
    \frametitle{Oder doch Hansa Rostock?}
    \begin{figure}
   \includegraphics[height = 6cm]{figures/hansa_logo.png}
\end{figure}
\end{frame}

\begin{frame}[mpidrbackground = 4]
    \frametitle{Rostock aus Sicht eines Demografen}
           \begin{figure}
               \centering
               \vspace{-10pt}
               \includegraphics[height=7.5cm]{figures/pop_pyramide_rostock.pdf}
           \end{figure}
    \end{frame}

\begin{frame}
    \frametitle{Messfehler?}
    \begin{figure}
    \vspace{-10pt}
   \includegraphics[height=5.5cm]{figures/rostock_ghsl.png}
   \caption{\tiny{Quelle: Europ\"aische Union (Copernicus): Global Human Settlement Layer}}
   \end{figure}
    
\end{frame}



\begin{frame}
\frametitle{Altersstruktur in Rostock}
    \begin{figure}
        \vspace{-10pt}
       \includegraphics[height=6.5cm]{figures/durchschnittsalter_district.pdf}
   \end{figure}
\end{frame}




\begin{frame}
    \frametitle{Auf Schrumpfkurs?}
    \begin{figure}
        \centering
        \includegraphics[width=0.8\linewidth]{figures/pop_proj_rostock.pdf}
        \caption{Caption}
        \label{fig:enter-label}
    \end{figure}
\end{frame}


\begin{frame}
    \frametitle{Bev\"olkerungsdynamik}
    \begin{equation}
        \textcolor{mpidr_red}{P_t} = \textcolor{mpidr_red}{P_{t-1}} + \textcolor{mpidr_blue}{\underbrace{B \ -\  D}_{\textrm{Nat\"urliche Bev\"olkerungsdynamik}}} + \textcolor{mpidr_grey}{\underbrace{I\  -\  E}_{\textrm{R\"aumliche Bev\"olkerungsdynamik}}}
    \end{equation}
\end{frame}


\begin{frame}
    \frametitle{Bevölkerungsveränderungen}
           \begin{figure}
               \centering
                \vspace{-10pt}
               \includegraphics[height=7cm]{figures/pop_change_districts_zoom.pdf}
           \end{figure}
\end{frame}

%====================================================

%====================================

%====================================
\begin{frame}[mpidrbackground=5]
    \frametitle{Dankesch\"on!}
    
    \vspace{0.8cm}
    
    \begin{columns}
          \column{0.5\textwidth}
          
            \setbeamercolor{normal text}{fg=mpidrgreen}
            \usebeamercolor[fg]{normal text}
           Henrik-Alexander Schubert\\ \vspace{0.3cm}
            \setbeamercolor{normal text}{fg=gray}
            Wissenschaftler \\ \vspace{0.15cm}
            \faIcon{at}schubert@demogr.mpg.de \\
            \usebeamercolor[fg]{normal text}

    
          \column{0.35\textwidth}



                \begin{figure}
                \centering
                \includegraphics[width=0.7\textwidth]{figures/qrcode_github.com.png}
                \caption{Link zu Analyse}
            \end{figure}
            %\faIcon{twitter}@27Henrik
             
       \end{columns}

\end{frame}

\appendix

\begin{frame}
\frametitle{Geschlechterver\"altnis}
    \begin{figure}
               \centering
               \vspace{-10pt}
               \includegraphics[height=7cm]{figures/sr_rostock.pdf}
           \end{figure}
\end{frame}


\begin{frame}
\frametitle{Bev\"olkerungsver\"anderung in den Stadtteilen}
           \begin{figure}
               \centering
                \vspace{-10pt}
               \includegraphics[height=7cm]{figures/popoulation_growth_barchart.pdf}
           \end{figure}
\end{frame}

% Events in the districts
\begin{frame}
\frametitle{Altersstruktur}
           \begin{figure}
               \centering
                \vspace{-10pt}
               \includegraphics[width=1\linewidth]{figures/altersstruktur_rostock.pdf}
           \end{figure}
\end{frame}

% Events in the districts
\begin{frame}
\frametitle{Demografische Ereignisse}
\begin{figure}
        \vspace{-10pt}
        \centering
       \includegraphics[height=7cm]{figures/events_districts.pdf}
   \end{figure}
\end{frame}


\begin{frame}
\frametitle{Mehr Frauen oder mehr M\"aenner?}
    \begin{figure}
    \vspace{-10pt}
       \includegraphics[height=6.5cm]{figures/sr_district_map.pdf}
   \end{figure}
\end{frame}


\begin{frame}
    \frametitle{Bevölkerungsveränderungen}
           \begin{figure}
               \centering
               \vspace{-10pt}
               \includegraphics[height=7cm]{figures/pop_change_districts.pdf}
           \end{figure}
\end{frame}

\begin{frame}
    \frametitle{Altersstruktur in den Bezirken}
           \begin{figure}
               \centering
               \vspace{-10pt}
               \includegraphics[height=7cm]{figures/altersstruktur_rostock_districts.pdf}
           \end{figure}
\end{frame}


\end{document}